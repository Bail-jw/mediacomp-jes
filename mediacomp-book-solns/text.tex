% text.tex, chapter of "Introduction to Media Computation"

\chapter{Creating and Modifying Text}

\begin{exercises}
\begin{ex}
You've just re-written the news. Who controls the past, controls the
future. Who controls the present, controls the past. 
\end{ex}

\begin{ex} %% 10.2
(a) Domain Name Server,
(b) FTP,
(c) URL,
(d) Web Server,
(e) HTTP,
(f) Client,
(g) HTML,
(h) this is the extra definition,
(i) IP.
\end{ex}

\begin{ex} %% 10.3
(a)	Basically the idea here is that anything that takes genuine human
judgement or insight cannot be done by writing a program.\newline
(b)	An array is a collection of data arranged in a systematic form. A
list and a matrix are examples of kinds of arrays. A matrix is a specific
kind of array that is two dimensional. A tree is a set of nested arrays.
Each "node" of a tree contains some data along with links to other "nodes".
\newline
(c)	Dot notation is used to call certain kinds of functions known as
methods. Methods are functions that apply directly to objects such as
strings or pictures. Dot notation usually eliminates the need for passing
in objects as input as they are called ON objects. For example
getWidth(picture) is equivalent to picture.getWidth()
\newline
(d)	The RGB values of a flesh-tone usually has very high red values. If
a person is slightly flushed or sunburned, their skin will be considered
part of the background. However if you are an alien or you are attempting
to use chromakey on non-red objects, then red is acceptable.
\newline
(e)	A function is a subroutine of instructions to accomplish a certain
task. A method is a type of function that applies its instructions to a
certain object. Methods are functions that cannot stand alone. An object
must be involved for a method to be executed.
\newline
(f) Files on a disk have a nested, hierarchical structure. A directory may
contain not just files, but other directories. They're to help you organize
your stuff! Arrays can't capture this structure easily (unless they
contained other arrays), but a tree structure is just right.
\newline
(g)	Consider a blue square against a red background. A
vector-graphics-based representation would say "Hey, that's a square on a
blank background! The width and height of the background are such and such
and the square is located at such and such coordinates with such and such
height". A format such as bitmap would say "The pixel at 1,1 is red. The
pixel at 1, 2 is red. The pixel at 1, 3 is red. The pixel at 1, 4 is
red..." Vector based graphics have a very significant advantage over bitmap
graphics when dealing with graphs, cartoonish-type pictures, etc. However,
the advantage is lost when photographs are considered, since there is no
set pattern for the color of the pixels that can be easily captured into
short equations.
\end{ex}

\begin{ex} %% 10.4
(a) 32 bits: 232 = 4,294,967,296 = 2564
\newline
(b) 16 bits: 216 = 65,536
\newline
(c) 24 bits: 224 = 16,777,216 = 2563
\newline
(d) 10 bits: 210 = 1,024
\end{ex}

\begin{ex} %% 10.5
(a)	The domain name server takes the human readable domain from a URL
(like www.google.com) and transforms it into an IP address, a form that the
computer can handle (64.233.161.99).
\newline
(b)	These are all protocols that are used to transfer different kinds
of data across the internet
\newline
FTP - File Transfer Protocol - used for passing files from computer to
computer
\newline
SMTP - Simple Mail Transfer Protocol - This is used to transfer emails
\newline
HTTP - Hyper Text Transfer Protocol - This is used to deliver webpages to
your browser.
\newline
(c)	Hypertext is the language that webpages are encoded in
\newline
(d)	A client is the computer that requests information from a server,
the computer that delivers the information.
\newline
(e)	When data is transferred on the internet, at some point it needs to
be converted into text so that it can be sent in the most simple fashion.
\newline
If you understand the raw protocols and encodings that data is stored in,
you can manipulate it into any form you want. For example, you could
manipulate the text on a webpage listing stock information and
automatically generate an image with a chart displaying the information.
\newline
(f)	The internet is a network of computers spread across the world. The
unique idea behind the creation of the internet was that there was no
central point controlling the internet as opposed to other systems that
were controlled by some central "brain".
\newline
(g)	An ISP is an internet service provider. Examples include AOL, MSN,
Mindspring, your campus' ResNet, etc.
\end{ex}

\begin{ex} %% 10.6
\begin{example}
def mirrorString(string):
  output = string[len(string) / 2]
  for x in range(0, len(string) / 2)
    character = string[len(string) / 2 - x - 1]
    output = character + output + character
  return output
\end{example}
\end{ex}

\begin{ex} %% 10.7
Do something creative! The answer to the problem is almost entirely in the
problem itself.
\end{ex}

\begin{ex} %% 10.8
\begin{example}
def percentageGenders(string):
  
  # These will be used to keep track of each tally
  males = 0
  females = 0
  
  # Go through each letter in the string
  for gender in string:
  
    # If the letter is M, add one to the male tally
    if gender.upper() == 'M':
      males = males + 1
    
    # If the letter is F, add one to the female tally
    if gender.upper() == 'F':
      females = females + 1
  
  # Figure out the total AS A FLOATING POINT DECIMAL
  total = males + females + 0.0
  
  # Determine the percentages
  mPercentage = str(100 * males / total) + '\%'
  fPercentage = str(100 * females / total) + '\%'
  
  # Present results
  print "There are " + mPercentage + " males and " + fPercentage + " females"
\end{example}
\end{ex}

\begin{ex} %% 10.9
\begin{example}
def fixItUp(string):
  # Which letters are messd up?
  theseAreBad = '7890uiohj'
  
  # And what are their CORRESPONDING correct letters?
  theseAreGood= 'uiopjklnm'
  
  # Each time we fix a letter, we'll
  # add it to the end of this string
  goodOutput = ''
  
  # Go through every letter in the document
  for letter in string:

    # Determine if the current letter is a messed up
    # letter, and if so, which one is it?  
    messUp = theseAreBad.find(letter)

    # If it was a messed up letter then...    
    if messUp != -1:
    
      # replace it with its corresponding correct letter
      letter = theseAreGood[messUp]
    
    # Add the correct letter to the end of the new document
    goodOutput = goodOutput + letter
    
  # Yay.
  return goodOutput
\end{example}
\end{ex}

\begin{ex} %% 10.10
\begin{example}
def doGraphics(list):

    # Fresh canvas
    canvas = makePicture(getMediaPath("640x480.jpg"))

    # Go through each command in the list
    for command in list:
  
        # b means make a pixel
        if command[0] == 'b':
    
            # Split command up into its separate words
            pieces = command.split(' ')
      
            # X coordinate and Y coordinate, 2nd and 3rd respectively
            x = int(pieces[1])
            y = int(pieces[2])
      
            # Draw it
            canvas.addRectFilled(black, x, y, 1, 1)
    
        # l means make a line
        if command[1] == 'l':
    
            # Split command up into its separate words
            pieces = command.split(' ')
      
            # Extracting X coordinates and Y coordinates
            x1 = int(pieces[1])
            y1 = int(pieces[2])
            x2 = int(pieces[3])
            y2 = int(pieces[4])
      
            # Draw it
            canvas.addLine(black, x1, y1, x2, y2)
  
    # voila!
    return canvas
\end{example}
\end{ex}

\end{exercises}
