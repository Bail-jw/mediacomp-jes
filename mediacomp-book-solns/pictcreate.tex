% pictcreate.tex, chapter of "Introduction to Media Computation"
\chapter[Combining Pictures]{Making Pictures by Combining Pieces}
    \graphicspath{{figs/}{figs/pictcreate/}{figs/pictencode/}}

\begin{exercises}
\begin{ex}
Try doing chromakey in a range---grab something out of its
background where the something is only in one part of a picture.
For example, put a halo around someone's head, but don't mess with
the rest of their body.
\end{ex}

\begin{ex}
Using the drawing tools presented here, draw a house---just go for
the simple child's house with one door, two windows, walls, and a
roof.
\end{ex}

\begin{ex}
Put a cabana on that beach.  Draw the house from the previous
exercise on the beach where we put the mysterious box previously.
\end{ex}

\begin{ex}
Now use your house to draw a town with dozens of houses at
different sizes. You'll probably want to modify your house
function to draw at an input coordinate, then change the
coordinate where each house is drawn.
\end{ex}

\begin{ex}
Draw a rainbow--use what you know about colors, pixels, and
drawing operations to draw a rainbow.  Is this easier to do with
our drawing functions or by manipulating individual pixels? Why?
\end{ex}

\begin{starex}
Write a function \code{doGraphics} that will take a list as input.
The function \code{doGraphics} will start by creating a canvas
from the \filename{640x480.jpg} file in the mediasources folder.
You will draw on the canvas according to the commands in the input
list.

Each element of the list will be a string. There will be two kinds
of strings in the list:

\begin{itemize}
\item ``b 200 120'' means to draw a black dot at x position 200
and y position 120--$(200,120)$. The numbers, of course, will
change, but the command will always be a ``b''. You can assume
that the input numbers will always have three digits.

\item ``l 000 010 100 200'' means to draw a line from position
$(0,10)$ to position $(100,200)$

\end{itemize}

So an input list might look like: \code{["b 100 200","b 101
200","b 102 200","l 102 200 102 300"]} (but have any number of
elements).
\end{starex}

\end{exercises}
