\chapter{Modifying Sounds using Loops}

\begin{exercises}
\begin{ex}
Try and see!
\end{ex}

\begin{ex}
Take a look and see! Then go read about the the physics of accoustics and
overtone series.
\end{ex}

\begin{ex}
Left as an exercise to the reader -- try playing with different sounds!
\end{ex}

\begin{ex}
\begin{example}
def increaseVolumeNamed(file):
  sound = makeSound(file)
  for sample in getSamples(sound):
    value = getSample(sample)
    setSample(sample, value * 2)
  play(sound)
\end{example}
\end{ex}

\begin{ex}
\begin{example}
def increaseVolumeNamed(inputFile, outputFile):
  sound = makeSound(inputFile)
  for sample in getSamples(sound):
    value = getSample(sample)
    setSample(sample, value * 2)
  writeSoundTo(sound, outputFile)
\end{example}
\end{ex}

\begin{ex}
\begin{example}
def increaseVolumeNamed(file, multiplier):
  sound = makeSound(file)
  for sample in getSamples(sound):
    value = getSample(sound):
    setSample(sample, value * multiplier)
\end{example}

Yes. There is no restriction on the magnitude of multiplier. If multiplier
is less than 1, then increaseVolumeNamed will actually decrease the volume.
\end{ex}

\begin{ex}
Drawing pictures is left as an exercise to the reader.
\end{ex}

\begin{ex}
The upper limit for allowed values of a sample is 32767 and the lower limit
is -32768. When you increase or decrease a sample value beyond that range,
it wraps around to the other side. This is similar to how color values
would wrap around to non-sensical values like if you tried to increase the
red in a picture of a person and it looked like you just gave them a
strange skin disease (see section 3.4). After multiplying the sample values
by a ratio greater than 1 for a while, the values become non-sensical which
eventually translates to static noises.
\end{ex}

\begin{ex}
The key idea to remember about making noise is that sound is caused by
oscillate vibrations. If you have a sound that oscillate 440 times per
second between 32767 and -32768, then that will be a relatively loud A
above middle C. However, if you have a large span of time filled with just
1 value that doesn't oscillate, then you will get silence, no matter how
large the magnitude of the sample values.
\end{ex}

\begin{ex}
The magnitude of the vibration is not the maximum sample value, but rather,
is the distance of the maximum sample value from the mean sample value.
Shifting a sound "vertically" does not affect the magnitude of the
vibration or the frequency, therefore there should be little noticeable
difference.
\end{ex}

\end{exercises}
