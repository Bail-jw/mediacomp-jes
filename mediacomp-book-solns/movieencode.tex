\chapter{Creating and Modifying Movies}
\begin{exercises}

\begin{ex} %% 12.1
\begin{example}
def animation(dir):
  spaceship = makePicture(getMediaPath("space\_ship.jpg"))
  planetColor = makeColor(200, 60, 0) # redish orange

  maxFrame = 50

  for frame in range(1, maxFrame + 1):
    # progress is a number between 0 and 1 denoting how far the animation
has progressed. 
    # It will be used to determine coordinates
    progress = (frame + 0.0) / maxFrame

    #you'll want a blank canvas for each frame
    canvas = makeEmptyPicture(320, 200)
		
    #draw a spaceship that flies from the left to the right and goes AROUND
the sun in the shape of a V
    spaceShipX = int(getWidth(canvas) * progress)
    if frame * 2 < maxFrame:
      spaceShipY = int(getHeight(canvas) * (.5 - progress))
    else:
      spaceShipY = int(getHeight(canvas) * (progress - .5))
    copy(spaceship, canvas, spaceShipX, spaceShipY)

    #draw a sun that will pulse from yellow to white. It will drift from
the right portion of the canvas to the left portion and slightly downward
    sunX = .75 - progress / 2
    sunY = .5 + progress / 8
    sunColor = makeColor(255, 255, 128 + int(127 * sin(frame * 2 * pi /
10)))
    canvas.addArcFilled(sunColor, int(sunX * getWidth(canvas)), int(sunY * getHeight(canvas)), 20, 20, 0, 360)

    #draw a planet that will orbit the sun 3 times during the course of the
animation
    x = int((cos(2 * pi * progress * 3) / 6 + sunX) * getWidth(canvas))
    y = int((cos(2 * pi * progress * 3) / 6 + sunY) * getHeight(canvas))
    canvas.addArcFilled(planetColor, x-10, y-10, 20, 20, 0, 360)

    # How many digits is the last frame?
    # How many digits is this frame?
    # Fill in the difference with 0's
    leadingZeros = "0" * (len(str(maxFrame)) - len(str(frame))) 
    writePictureTo(canvas, dir + '//frame' + leadingZeros + str(frame) + '.jpg') 

def copy(pic, canvas, xoffset, yoffset):
  bgcolor = getColor(getPixel(pic, 1, 1))
  for p in getPixels(pic):
    if 1 <= getX(p) <= getWidth(canvas) and 1 <= getY(p) <= getHeight(canvas):
      color = getColor(p)
      if distance(color, bgcolor) > 20):
        targetX = getX(p) + xoffset
        targetY = getY(p) + yoffset
        targetPixel = getPixel(canvas, targetX, targetY)
        setColor(targetPixel, color)
\end{example}
\end{ex}

\begin{ex} %% 12.2
/* I believe the ABC news site has drastically changed since 
the time you wrote this question. I was thinking I could
throw together an exerpt of a "fake" news page and show
how one would parse that. Is that ok? */
\end{ex}

\begin{ex} %% 12.3
Rather than passing the jungle picture in to the jungle picture in as the
third parameter of the swapbg function, simply pass the blank wall in
twice.
\end{ex}

\begin{ex} %% 12.4
\begin{example}
# Assuming that both are the same size...

def blend(dir, pictureFileA, pictureFileB):

  # Refer to fadeFrom and fadeInto to determine your color values
  fadeFrom = makePicture(pictureFileA)
  fadeInto = makePicture(pictureFileB)

  # All we basically need here is a picture to use as the frame
  # which must be the same size as the other two picture
  canvas = makePicture(pictureFileA)

  maxFrame = 100

  for frame in range(maxFrame):

    # Determine a ratio of how much of each picture you want
    # in this frame. 
    ratio = frame / (maxFrame - 1.0)
  
    for x in range(1, getWidth(canvas) + 1):
      for y in range(1, getHeight(canvas) + 1):

        # Get the pixels you'll be reading from
        pixelA = getPixel(fadeFrom, x, y)
        pixelB = getPixel(fadeInto, x, y)

        # Get the pixel you'll be writing to
        target = getPixel(canvas, x, y)

        # Calculate and set the red value
        redA = getRed(pixelA) * (1 - ratio)
        redB = getRed(pixelB) * ratio
        setRed(target, redA + redB)

        # Calculate and set the green value
        greenA = getGreen(pixelA) * (1 - ratio)
        greenB = getGreen(pixelB) * ratio
        setGreen(target, greenA + greenB)

        # Calculate and set the blue value
        blueA = getBlue(pixelA) * (1 - ratio)
        blueB = getBlue(pixelB) * ratio
        setBlue(target, blueA + blueB)

    # Convert the numerical frame number to a string
    framenum = str(frame)

    # Determine how many 0's should be in front of the frame number
    zeros = "0" * (len(str(maxFrame)) - len(framenum))

    # Concatenate the Zeros onto the front of the frame number
    framenum = zeros + framenum
  
    # Write the picture to the directory with the proper file name
    writePictureTo(canvas, dir + '//' + 'frame' + framenum + '.jpg')
\end{example}
\end{ex}

\end{exercises}
