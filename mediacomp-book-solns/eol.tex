% text.tex, chapter of "Introduction to Media Computation"

	%\graphicspath{{figs/}{figs/pictcreate/}}

Manipulating text is what this chapter is about.  It's very important for us because a very common form of text for us today is \newterm{HyperText Markup Language (HTML)}.


\section{A recipe to generate HTML}

	Be sure to do \code{ setMediaFolder()} before running this!

\begin{recipe}{A recipe to generate HTML}
\begin{example}
def html():

    # To use this routine the Media folder must be used for your output and picture files

    myHTML =  getMediaPath("myHTML.html")
    pictureFile = getMediaPath("barbara.jpg") 
    eol=chr(11) #End-of-line character                 
    # The following line builds a literal string that includes both single and double quotes
    buildSpecial = "<IMG SRC=\""+ pictureFile + '" ALT= "I am one heck of a programmer!">'+eol 
   
    outFile = open(myHTML,'w')
      
    outFile.write( '<HTML>'+eol)
    outFile.write('<HEAD>'+eol)
    outFile.write('<TITLE>Homepage of Georgia P. Burdell</TITLE>'+eol)
    outFile.write('<LINK REL=STYLESHEET TYPE="text/css" HREF="style.css">')      
    outFile.write('</HEAD>'+eol)
    outFile.write('<BODY>'+eol)
    outFile.write('<CENTER><H2>Welcome to the home page of Georgia P. Burdell!</H2></CENTER>'+eol)
    outFile.write('<BR>'+eol)
    outFile.write('<P> Hello, and welcome to my home page! As you should have already'+eol)
    outFile.write(' guessed, my name is Georgia, and I am a <A HREF=http://www.cc.gatech.edu><B>'+eol) 
    outFile.write('Computer Science</B></A> major at <A HREF=http://www.gatech.edu><B>'+eol)
    outFile.write('Georgia Tech</B> </A>'+eol)
    outFile.write('<BR>'+eol)
    outFile.write('Here is a picture of me in case you were wondering what I looked like.'+eol)
    outFile.write('</P>'+eol)
    outFile.write(buildSpecial)                                        # Write the special line we built up near the top
    outFile.write('<P><H4> Well, welcome to my web page. The majority of it is still under construction, so I don\'t have a lot to show you right now. '+eol)
    outFile.write('I am in my 75th year at Georgia Tech but am taking CS 1315 so I don\'t have a lot of spare time to update the page.'+eol)
    outFile.write('I promise to start real soon!'+eol)
    outFile.write('--Georgia P. Burdell</P></H4>'+eol)
    outFile.write('<HR>'+eol)
    outFile.write('<PIf you want to send me e-mail, click <><A HREF = "mailto:name@ece.gatech.edu">name@ece.gatech.edu</A>'+eol)
    outFile.write('<HR></P>'+eol)
    outFile.write('<CENTER><A HREF="http://www.cc.gatech.edu/">'+eol)
    outFile.write('<IMG SRC="http://www.cc.gatech.edu/newhome_images/CoC_logo_anniv2.gif" ALT= "To my school"></CENTER>'+eol)
    outFile.write('</A>'+eol)
    outFile.write('</BODY>'+eol)
    outFile.write('</HTML>'+eol)

    outFile.close()
\end{example}
\end{recipe}


\section{Converting from sound to text to graphics}

	The creation of a signal visualization of a sound is something that Excel can do, too.

\begin{recipe}{Convert a sound into a text file that Excel can read}
\begin{example}
def writeSampleValue():
    f = pickAFile()					# File where the original sound resides
    source = makeSound(f)
    eol = chr(11)

    getPath = f.split('.')                # find the '.' in the full file name
    suffix = '.txt'
    myFile = getPath[0] + suffix          # get the part leading up to the '.' and make it  a ".txt" file
    outFile = open(myFile,'w')
      
    endCurrentSound = getLength(source)
    
    for pos in range (1, endCurrentSound+1): 
        sampVal =  getSampleValueAt(source,pos) 
        stringVal = '%s' % (sampVal)
        outFile.write(stringVal + eol)
        #outFile.write(eol)
    
    outFile.close()
\end{example}
\end{recipe}


