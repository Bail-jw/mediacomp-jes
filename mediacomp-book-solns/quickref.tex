

\chapter{Quick Reference to Python}



% quickref.tex, chapter of "Introduction to Media Computation"



    \graphicspath{{figs/}{figs/cs/}}



\section{Variables}



Variables start with a letter and can be any word \emph{except}

the \newterm{reserved words}.  Those are:



\begin{quote}

and, assert, break, class, continue, def, del, elif, else, except,

exec, finally, for, from, global, if, import, in, is, lambda, not,

or, pass, print, raise, return, try, while, yield

\end{quote}



We can use \code{print} to get a printable representation of an

expression (such as a variable).  If we simply type the variable,

without \code{print}, we get the internal representation:

functions and objects tell us about where they are in memory, and

strings appear with their quotes.



\begin{verbatim}

>>> x = 10

>>> print x

10

>>> x

10

>>> y='string'

>>> print y

string

>>> y

'string'

>>> p=makePicture(pickAFile())

>>> print p

Picture, filename C:\Documents and Settings\Mark Guzdial\My

Documents\mediasources\7inX95in.jpg height 684 width 504

>>> p

<media.Picture instance at 6436242>

>>> print sin(12)

-0.5365729180004349

>>> sin

<java function sin at 26510058>

\end{verbatim}

