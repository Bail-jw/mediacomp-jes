% introprogram.tex, chapter of "Introduction to Media Computation"
\chapter{Introduction to Programming}
\begin{exercises}

\begin{ex}
\begin{itemize}
\item An algorithm is a general yet systematic method for accomplishing
some sort of task. An algorithm itself is independent of any programming
language, but when an algorithm has been written as a program, then it
becomes an "implementation of an algorithm".

\item Hierarchical decomposition: not defined yet.

\item Comments allow you to keep note of what was going through your mind
while you were writing code, or perhaps a run-down of how your code works
to someone else reading it. Reading code is harder than writing code.

\item When data is stored on the computer, it eventually has to be stored
as 1's and 0's. When you want to save a picture, you don't have the luxury
of saving a literal picture on tape like with old film cameras. The media
has to be "encoded" somehow so that it can be saved in a form that the
computer understands. The method that data is broken down into numerical
components is called an "encoding".

\item Moore's Law states that every 18 months, the number of transistors
will double for the same cost. In other words, the same amount of money can
buy you twice the computing power every 18 months. This trend was first
observed by Gordon Moore, one of the founders of Intel in 1965 and has held
true even through modern times.
\end{itemize}
\end{ex}

\begin{ex}
\begin{itemize}
\item
A def means you are defining a new function. You are adding a new word to
 JES's vocabulary.

In this statement you are adding a new function to JES's vocabulary. This
new function called "someFunction" will take in two values. The code for
this function will be directly below this line indented.

\item Print converts the data that follows it into a meaningful textual
representation and displays it on the output screen. \code{print a} does
something that depends on what a is. If a is storing a numerical or string
value, then the raw contents will be displayed on the output screen. If a
is storing an object, then useful information about the object will most
likely be displayed. For example if a was a picture, print a would inform
you that it is a picture object and give the width and the height.

\item If p is a valid image, then it will display a window revealing the
image.  If the window is already displayed, then show(p) will refresh the
contents of the window. However if you were to call show on a different
picture, for example show(q), then a separate window will be generated.
\end{itemize}
\end{ex}

\begin{ex}
Multiplying a number by a string will return that string n times where n is
the number you multiplied by.

Multiplying a string by a string will make the universe collapse...so don't
do that.
\end{ex}


\begin{ex}
No explanation required, really.
\end{ex}


\begin{ex}
Just like pictures and sounds, functions deep down are objects too.  When
you end a function name with a set of parenthesis, this indicates that you
want to run the function, however without the parenthesis, you are merely
handling the function as though it were just another piece of data. When
you print the function name, it tells you that it is an instance of a
function object and where it is located in memory.
\end{ex}

\end{exercises}
