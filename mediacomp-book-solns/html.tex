\chapter{Making Text for the Web}

% html.tex, chapter of "Introduction to Media Computation"

\begin{exercises}

\begin{ex}%% 11.1:
(a) The company stores information from it's website in a database.
When you go to the website, the information is taken from the database and
placed in the html so that it appears on the site and you can see it.
Things like pictures, or news articles, or products that can be sold by the
company are types of things that are stored in the database.  When the
database is broken, the html cannot be created and so the website will not
work
\newline

(b)  The html for the website is created by a program. The program is
designed to interface with the database. If contains the structure for the
format of the site, and has functions that take the information from the
database and insert them into the html tags so that the resultant html file
contains all the correct formating and tags with all the correct and up to
date information in it. The program only needs to be written once and can
run over and over again to update the page without having to rely on people
to type in all the html since the risk of errors increases when a person
has to type all the html by hand.
\end{ex}

\begin{ex}%% 11.2:
The Domain Name Server (DNS) is like an internet address book where your
brower (Internet Explorer, Netscape, Safari, Mozilla) can send
in the url (Universal Resource Locator) the http://www.cnn.com and search
for the IP Address 64.236.24.20. Then the IP address is used
to locate the website and display it in your internet browser. The url
alone cannot be used to find the site. When the DNS is not
working, your internet browser cannot locate the IP address and cannot find
the site. However if you provide the IP address, there is
no need for the DNS and your internet browser can go directly to the site.
\end{ex}

\begin{ex}%% 11.3:
\begin{example}
### needs to be written.

import os

def pictureLinks(directory):
    file = open (directory + "//index.html", "wt")
    file.write(doctype())
    file.write(title("Picture Links"))
    string = ""
    for file in os.listdir(directory):
    if file.lower().endswith(".jpg"):
    string = string + ' \\n'
    file.write(body(string))
    file.close()

def doctype():
    return ''

def title(titlestring):
    return """
def body(bodystring): return ""+bodystring+""
\end{example}
\end{ex}

\begin{ex}%% 11.4:
\begin{example}
import random
def weatherComment(temp):
    coldComments = ["Watch out for ice!',"Is it going to snow?"]
    coolComments = ["I can't wait for winter to be over!","Come on, Spring!"]
    warmComments = ["It's getting warmer!","Light jacket weather."]
    hotComments = ["FINALLY! Summer!","Time to go swimming!"]

    if temp 32:
        return random.choice(coldComments)
    if temp >= 32 and temp = 50:
        return random.choice(coolComments)
    if temp > 50 and temp 80:
        return random.choice(warmComments)
    if temp >= 80:
        return random.choice(hotComments)
\end{example}
\end{ex}

\end{exercises}
